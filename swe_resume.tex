\documentclass[]{resume-openfont}

\pagestyle{fancy}
\resetHeaderAndFooter

%--------------------------------------------------------------
% Convenience command - make it easy to fill template

% Create job position command. Parameters: company, position, location, when
\newcommand{\resumeHeading}[4]{\runsubsection{\uppercase{#1}}\descript{ | #2}\hfill\location{#3 | #4}\fakeNewLine}

% Create education heading. Parameters: Name, major, year, where, when, minor(s)
\newcommand{\educationHeading}[4]{\runsubsection{#1}\hspace*{\fill} \location{#3  #4}\\
\runSchoolName{#2}\vspace{1pt}\fakeNewLine}

% Create project heading. Parameters: Name, link, Tech stack
\newcommand{\projectHeading}[2]{\Project{#1}\hfill{#2}\\}

\newcommand{\projectHeadingWithDate}[3]{\Project{#1}\hfill{#2}
\descript{#3}\\}

% Parameters: courses
\newcommand{\courseWork}[9]{\textbf{\underline{Relevant Coursework:}} \vspace{2pt}\newline
\\ #1; #2; #3; #4; #5; #6; #7; #8; #9}
\newcommand{\course}[1]{#1}

% Awards heading
\newcommand{\awardsHeading}[1]{\runsubsection{#1}}

% Parameters: awards
\newcommand{\awards}[9]{\runsubsection{#1} \hspace*{378} \runsubsection{#2} \\
{#3} \hspace*{250} {\raggedleft #6} \\
{#4} \hspace*{288pt} {#7} \\
{#5} \hspace*{358pt} {#8} \\
 \hspace*{\fill} {#9}}
%--------------------------------------------------------------
%     Profile
%--------------------------------------------------------------
\def\name{Austin Barton} % Name Here
\def\phone{(404) 593-7833}
\def\email{barton.austin.t@gmail.com}
\def\studentemail{abarton40@gatech.edu}
\def\LinkedIn{austintbarton} % linkedin.com/in/______
\def\studentprof{austintbarton} % github username
\def\role{} % JOB TITLE
%--------------------------------------------------------------
 %
\RequirePackage{fancyhdr}
\fancypagestyle{plain}{%
\fancyhf{}
\lhead{\phone \\ % PHONE
	    \href{mailto:\email}{\email} \\
	    \href{mailto:\studentemail}{\studentemail}}% EMAIL}
	\chead{%
	    \centering {\Huge \textbf{\name}}}
    	%\rhead{\href{https://gatech-csm.symplicity.com/profiles/austin.barton\studentprof}{Student Professional Profile \studentprof} \\% Student Profile
             \rhead{\href{https://github.com/abarton51}{github.com/abarton51} \\% Student Profile
	    \href{https://www.linkedin.com/in/austintbarton}{linkedin.com/in/\LinkedIn}} % LinkedIn
\renewcommand{\headrulewidth}{1.75pt}% 2pt header rule
\renewcommand{\headrule}{\hbox to\headwidth{%
  \leaders\hrule height \headrulewidth\hfill}}
}
\pagestyle{plain}

\setlength{\headheight}{70pt}
\setlength{\headsep}{0pt}

\makeatletter
\let\ps@empty\ps@plain
\let\ps@firstpage\ps@plain
\makeatother
%--------------------------------------------------------------
\begin{document}
%--------------------------------------------------------------
%     Profile
%--------------------------------------------------------------
%--------------------------------------------------------------
%     Education
%--------------------------------------------------------------
\section{Education}
\vspace{-4pt}
\educationHeading{\large{Georgia Institute of Technology}}{Bachelor of Science in Mathematics and Computer Science (GPA: 3.69)}{}{Graduation Date: May 2025}
%   Coursework
\courseWork{Natural Language Processing}{Deep Learning}{Machine Learning}{Information Security}{Algorithm Analysis}{Data Structures and Algorithms}{Objects and Design}{Information Security}{Database Systems}{; Systems and Networks}{; Statistical Theory}{; Robotics}{; Information Theory}{; Intro to A.I.}{; Computer Organization}{; Probability Theory}

\sectionsep
\vspace{-12pt}

\section{Skills}
\vspace{-6pt}
%--------------------------------------------------------------
%     Skills
%--------------------------------------------------------------
\begin{skillList}
    \singleItem{Programming Languages:}{\textit{Proficient:} Python, Java, SQL, C. \textit{ Familiar:} Rust, C++, TypeScript, Assembly (x86), Bash, R.}\\
    \singleItem{Libraries, Frameworks, Etc.:}{yTorch, Scikit-learn, NumPy, Pandas, PySpark, Matplotlib, SciPy, Keras, LangChain, Seaborn.}\\
    \singleItem{Cloud Computing:}{Amazon/AWS - DynamoDB, Lambda, S3, Glue, IAM, Cloudwatch, Redshift, Step Functions, AWS CDK.}\\
    \singleItem{Data Science:}{Deep Learning, Machine Learning, Natural Language Processing, RAG, Computer Vision, Statistics, Generative AI.}\\
    \singleItem{Computer Systems:}{GNU/Linux, Memory Hierarchy, Low-level debugging (GDB), Build Systems (Make, Cargo), Network Protocols.}\\
    \singleItem{Misc.:}{VCS (Git), Containerization (Docker), Computer Systems, Databases, Data Structures, Algorithms, OOP/OOD, Agile, CI/CD.}
\end{skillList}

\sectionsep
\vspace{-10pt}

%--------------------------------------------------------------
%     Work Experience
%--------------------------------------------------------------
\section{Work Experience}
\vspace{-2pt}
\resumeHeading{Amazon Web Services (AWS)}{Software Engineer Intern}{Bellevue, WA}{May 2024 - Aug 2024}
\begin{bullets}
    %\item Designed, implemented, and deployed to production an ETL data flow and manipulation augmentation upon an existing Business Intelligence service to increase the reliability and accuracy of business metrics by accessing, processing, and analyzing a significant portion (~35\%) of previously unaccounted customer scenarios.
    \item Designed, implemented, and deployed an ETL (Extract, Transform, Load) data integration process for the Business Intelligence of an AWS service using cloud infrastructure for automated workflows, enhancing the reliability and accuracy of business metrics by integrating ~35\% of previously unaccounted customer scenarios. %Leveraged Redshift for optimized SQL queries and automated data ingestion workflows, leading to improved data-driven decision-making across the organization.
    \item Implemented a robust dependency mocking framework, enabling the simulation of Coral service dependencies with dynamically generated test data, leading to more comprehensive integration tests.
    %\item Designed, developed, and deployed integration tests covering a multitude of scenarios for Lambda Functions.
    %Implemented dependency mocking utilizing self-hosted AWS DynamoDB tables deployed through AWS Cloud Development Kit (CDK) in Typescript.
    \item Leveraged AWS Cloud Development Kit (CDK) to automate the provisioning, deployment, and management of cloud-based resources and services, streamlining infrastructure processes.
    %\item Actively participated in Agile development practices, including daily SCRUM meetings and bi-weekly sprints.
    %\item Designed, developed, and deployed integration tests with dependency mocking through Amazon Pipelines CI/CD automated workflow service covering a multitude of scenarios for AWS Lambda Functions.
\end{bullets}
\vspace{2pt}
\resumeHeading{United States Marine Corps}{Infantry Assaultman (E-5)}{Camp Lejeune, NC}{Oct 2016 - Oct 2020}
\begin{bullets}
    \item Led, mentored, and collaborated with a team of 12 Marines to prioritize mission accomplishment under hazardous working conditions in support of riflemen for numerous operations and two deployments.
\end{bullets}

\sectionsep
\vspace{-12pt}

%--------------------------------------------------------------
%     Research Experience
%--------------------------------------------------------------
\section{Research Experience}
\vspace{-2pt}
\resumeHeading{\href{https://www.vip.gatech.edu/teams/vxv}{Vertically Integrated Program (VIP)}}{Student Researcher}{Atlanta, GA}{January 2024 - May 2024}
\begin{bullets}
    \item Researching datasets for benchmarking LLMs' abilities to identify SEC violations given a scenario description.
\end{bullets}
\vspace{0pt}
\resumeHeading{\href{https://github.com/abarton51/BINNs_EQL_Covasim}{Research Experience for Undergraduates}}{Researcher}{Raleigh, NC}{May 2023 - Aug 2023}
\begin{bullets}
    \item Researched parameter estimation and modeling at the NSF and NSA sponsored research program hosted by NC State University. Implemented Physics-Informed Neural Networks and equation learning techniques in Python using PyTorch to infer a system of differential equations for an agent-based model with adaptive behavior.
\end{bullets}

\sectionsep
\vspace{-12pt}

%--------------------------------------------------------------
%     Projects
%--------------------------------------------------------------
\section{Projects}
\href{https://github.com/abarton51/MambaRALM}{\projectHeadingWithDate{Mamba vs Transformer based RALMs}{CS 4650, NLP, Georgia Tech |}{Jan 2024 - May 2024}{}{}}
Analyzed performance of retrieval augmented language models (RALMs) with Mamba and Transformer based architectures for knowledge intensive tasks over increasing number of retrieved chunks.
\smallskip

\href{https://github.com/abarton51/Hyperspectral-Deep-Deconvolution}{\projectHeadingWithDate{Single-shot Hyperspectral Deep Deconvolution}{CS 4644, Deep Learning, Georgia Tech |}{Aug 2023 - Dec 2023}{}{}}
Enhanced hyperspectral images by mitigating distortions inherent in snapshot acquisitions by leveraging blind deconvolution with a U-Net. Demonstrated models capable of deblurring and restoring spectral information.
\smallskip

\href{https://abarton51.github.io/CS_4641_Project/tabs/final_report.html}{\projectHeadingWithDate{Exploring Music Classification}{CS 4641, Machine Learning, Georgia Tech |}{Aug 2023 - Dec 2023}{}{}}
Led a group project on exploring methods in music classification over two distinct datasets. Created a framework in Python for audio data processing, dimensionality reduction, and various supervised learning methods.
\smallskip

\href{https://github.com/abarton51/cnn-bird-clf/tree/main}{\projectHeadingWithDate{Bird classification with CNNs}{MATH 4210, Math of Data Science, Georgia Tech |}{Jan. 2023 - May 2023}}
Built and trained multiple CNN models in Python which resulted in achieving up to a 95\% test accuracy on an image classification task for a dataset consisting of 88,000 images of 515 distinct bird species.
\smallskip

%\href{https://github.com/abarton51/simplerw_and_pathenum}{\projectHeadingWithDate{Simple Random Walks and Enumeration}{MATH 3235, Probability Theory, Georgia Tech |}{Oct. 2022 - Dec. 2022}}
%\smallskip

%\projectHeadingWithDate{Dyck path enumeration}{Georgia Tech School of Mathematics |}{Sep. 2022 - Dec. 2022}
%Participated in the Directed Reading Program hosted by the School of Math at Georgia Institute of Technology. I studied enumerative combinatorics related to Dyck paths under the guidance of a postdoctoral researcher.
%\smallskip


%\projectHeadingWithDate{Warrior Scholar Project}{Warrior Scholar Project at Yale University |}{August 2020}
%College education preparation program for enlisted veterans transitioning to student life hosted by Yale University.
\sectionsep
\vspace{-14pt}

%\section{Conferences and Presentations}
%\vspace{-4pt}
%\begin{skillList}
 %   \singleItem{BEER 2023}{Attended the 2023 International Symposium on Biomathematics and Ecology Education and Research as the lead} \\
  %  \singleItem{}{speaker for an oral research presentation.} \\
   % \singleItem{2023 NCSU Undergraduate Research Symposium}{Led a poster presentation session for research done with an REU program.}
%\end{skillList}
\iffalse
\section{Activities}
\vspace{-6pt}
\begin{skillList}
    %\singleItem{Data Science @ GT}{Member of KaggleCLEF team. We compete in Kaggle competitions and practice and learn about data science.} \\
    \singleItem{Directed Reading Program}{Studied enumerative combinatorics under the guidance of a postdoctoral researcher over Fall 2022.} \\
    %\singleItem{Georgia Tech Cycling Club}{Recreational mountain biking and road cycling.}
\end{skillList}
\fi
\iffalse
%--------------------------------------------------------------
%     Skills
%--------------------------------------------------------------
\section{Skills}
\begin{skillList}
    %\singleItem{Mathematics:}{Probability and Statistics, LaTeX}\\
    \singleItem{Programming Languages:}{Python, Java, C}\\
    \singleItem{Programming Packages, Libraries, etc.}{Pandas, TensorFlow, Keras, Numpy, Matplotlib, Seaborn} \\
    \singleItem{Programming Concepts}{Artificial Intelligence, Data Science, Data structures, Algorithms, Object-oriented programming}\\
    \singleItem{Other:}{Enthusiasm; Collaboration; Initiative; Communication; Analytical}
\end{skillList}
\fi
\vspace{-2pt}
%this needs some editing but overall not bad
\iffalse
\section{Security Clearance}
\singleItem{Secret-level clearance (2016 - 2020),}{currently inactive}
\fi
\iffalse

\vspace{1pt}

\section{Awards}
\awards{Academic Awards}{Notable Military Awards}{Dean's List \hfill John H. and Susan S. Traendly Scholar \\ Zell Miller Scholar}{Certificate of Commendation \hfill Meritorious Promotion \\ Good Conduct Medal \hfill Arctic Service Medal}
\fi
\vspace{2pt}
\section{Awards}
\vspace{-2pt}
\awards{Academic Awards}{Military Awards}%
{Edith Nourse Rogers STEM Scholar}%
{John and Susan Traendly Scholar}%
{Zell Miller Scholar}%
{Certificate of Commendation}%
{Meritorious Promotion}%
{Good Conduct Medal}%

\end{document}