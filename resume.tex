\documentclass[]{resume-openfont}

\pagestyle{fancy}
\resetHeaderAndFooter

%--------------------------------------------------------------
% Convenience command - make it easy to fill template

% Create job position command. Parameters: company, position, location, when
\newcommand{\resumeHeading}[4]{\runsubsection{\uppercase{#1}}\descript{ | #2}\hfill\location{#3 | #4}\fakeNewLine}

% Create education heading. Parameters: Name, major, year, where, when, minor(s)
\newcommand{\educationHeading}[5]{\runsubsection{#1}\hspace*{\fill} \location{#3  #5 | #4}\\
\runSchoolName{#2}\vspace{1pt}\fakeNewLine}

% Create project heading. Parameters: Name, link, Tech stack
\newcommand{\projectHeading}[2]{\Project{#1}\hfill{#2}\\}

\newcommand{\projectHeadingWithDate}[3]{\Project{#1}\hfill{#2}
\descript{#3}\\}

% Parameters: courses
\newcommand{\courseWork}[9]{\textbf{\underline{Relevant Upper Coursework:}} \vspace{2pt}\newline
\\ #1; #2; #3; #4; #5; #6; #7; #8; #9}
\newcommand{\course}[1]{#1}

% Parameters: courses
\newcommand{\student}[1]{\textbf{Teacher Assistant (TA):} #1}

% Awards heading
\newcommand{\awardsHeading}[1]{\runsubsection{#1}}

% Parameters: awards
\newcommand{\awards}[9]{\runsubsection{#1} \hspace*{325} \runsubsection{#2} \\
{#3} \hspace*{264} {\raggedleft #6} \\
{#4} \hspace*{326pt} {#7} \\
{#5} \hspace*{358pt} {#8} \\
 \hspace*{\fill} {#9}}
%--------------------------------------------------------------
%     Profile
%--------------------------------------------------------------
\def\name{Austin Barton} % Name Here
\def\phone{(704) 438-7616}
\def\email{barton.austin.t@gmail.com}
\def\studentemail{abarton40@gatech.edu}
\def\LinkedIn{austintbarton} % linkedin.com/in/______
\def\studentprof{austintbarton} % github username
\def\role{} % JOB TITLE
%--------------------------------------------------------------
 %
\RequirePackage{fancyhdr}
\fancypagestyle{plain}{%
\fancyhf{}
\lhead{\phone \\ % PHONE
	    \href{mailto:\email}{\email} \\
	    \href{mailto:\studentemail}{\studentemail}}% EMAIL}
	\chead{%
	    \centering {\Huge \textbf{\name}}}
    	%\rhead{\href{https://gatech-csm.symplicity.com/profiles/austin.barton\studentprof}{Student Professional Profile \studentprof} \\% Student Profile
             \rhead{\href{https://github.com/abarton51}{github.com/abarton51} \\% Student Profile
	    \href{https://www.linkedin.com/in/austintbarton}{linkedin.com/in/\LinkedIn}} % LinkedIn
\renewcommand{\headrulewidth}{1.75pt}% 2pt header rule
\renewcommand{\headrule}{\hbox to\headwidth{%
  \leaders\hrule height \headrulewidth\hfill}}
}
\pagestyle{plain}

\setlength{\headheight}{60pt}
\setlength{\headsep}{0pt}

\makeatletter
\let\ps@empty\ps@plain
\let\ps@firstpage\ps@plain
\makeatother
%--------------------------------------------------------------
\begin{document}
%--------------------------------------------------------------
%     Profile
%--------------------------------------------------------------
%--------------------------------------------------------------
%     Education
%--------------------------------------------------------------
\section{Education}
% Put school first and degree second if your school is reputable
\educationHeading{\large{Georgia Institute of Technology}}{B.S. in Mathematics and Computer Science (GPA: 3.77)}{}{Atlanta, GA}{Projected Graduation Date: May 2025}
%   Coursework
\courseWork{Deep Learning}{Machine Learning}{Statistical Theory}{Mathematics of Data Science}{Database Systems}{Information Theory}{Artificial Intelligence}{Probability Theory}{Algorithm Analysis}{; Data Structures and Algorithms}%\course{; Real Analysis I}\course{; Second Course in Linear Algebra}\course{; Data Structures and Algorithms}

\sectionsep
\vspace{-12pt}

\section{Skills}
\vspace{-6pt}
%--------------------------------------------------------------
%     Skills
%--------------------------------------------------------------
\begin{skillList}
    %\singleItem{Mathematics:}{Probability and Statistics, LaTeX}\\
    \singleItem{Programming Languages:}{\textit{Highly Proficient} in Python. \textit{Proficient} in SQL, Java, LaTeX. \textit{Familiar} with MATLAB, C.}\\
    \singleItem{Libraries, Frameworks, Etc.:}{PyTorch, Scikit-learn, NumPy, Keras, Pandas, SciPy, Matplotlib, Seaborn, Git.}\\
    %\singleItem{Software Technology:}{Pandas, TensorFlow, Numpy, Seaborn, Matplotlib}\\
    %\singleItem{Programming Packages, Libraries, etc.}{Pandas, TensorFlow, Keras, Numpy, Matplotlib, Seaborn} \\
    \singleItem{Concepts:}{Deep Learning, Unsupervised and Supervised Learning, Generative Models, Seq2seq Models, Computer Vision,}\\
    \singleItem{}{Statistics, Data Science, AI, Databases, Data Structures, Algorithms, OOP, VCS, Agile Methodology}
    %\singleItem{Other:}{Passion, Analytical, Collaboration, Enthusiasm, Initiative, Adaptability}
\end{skillList}

\sectionsep
\vspace{-10pt}

%--------------------------------------------------------------
%     Research Experience
%--------------------------------------------------------------
\section{Research Experience}
\vspace{-2pt}
\resumeHeading{\href{https://github.com/abarton51/BINNs_EQL_Covasim}{Research Experience for Undergraduates}}{Researcher}{Raleigh, NC}{May 2023 - Aug 2023}
\begin{bullets}
    \item Attended the NSF and NSA sponsored research program (REU) at the Dept. of Math at NC State University. 
    \item Led research on \href{https://github.com/abarton51/BINNs_EQL_Covasim}{parameter estimation} using informed neural networks and equation learning to obtain closed-form equations for an ODE approximation of an agent-based model with added adaptive behaviors. 
    %\item{Developed a computational pipeline in Python} which implements custom informed neural networks using PyTorch and applies sparse regression methods to obtain closed-form equations for an ODE approximation.
    \item Awarded Best Poster Presentation for the Math and Statistics REU program.
   % \item  \href{https://github.com/abarton51/BINNs_EQL_Covasim}{code} is implemented in Python using PyTorch and interacts with the \href{https://github.com/InstituteforDiseaseModeling/covasim}{Covasim} simulator for data generation.
\end{bullets}

\sectionsep
\vspace{-12pt}

%--------------------------------------------------------------
%     Work Experience
%--------------------------------------------------------------
\section{Work Experience}
\vspace{-4pt}

\resumeHeading{United States Marine Corps}{Infantry Assaultman (E-5)}{Camp Lejeune, NC}{Oct 2016 - Oct 2020}
\begin{bullets}
    \item Served 4 years active duty and honorably separated as a Sergeant/E-5. 
    \item Led, mentored, and collaborated with a team of 12 Marines to prioritize mission accomplishment under hazardous working conditions and stressful environments as part of Weapons Plt., Fox Co., 2/6 Marines. 
    \item Acted as the senior representative of the Assault section for numerous training operations and two overseas deployments for a total of 15 months deployed.
\end{bullets}

\sectionsep
\vspace{-12pt}

%--------------------------------------------------------------
%     Projects
%--------------------------------------------------------------
\section{Projects}
\href{https://github.com/abarton51/Hyperspectral-Deep-Deconvolution}{\projectHeadingWithDate{Single-shot Hyperspectral Deep Deconvolution}{CS 4644, Deep Learning, Georgia Tech |}{Aug 2023 - Dec 2023}{}{}}
Course project on Single-shot Hyperspectral Deep Deconvolution. Our proposed method aims to enhance the quality of hyperspectral images by mitigating distortions inherent in snapshot acquisitions by leveraging a blind deconvolution approach with a U-Net neural network architecture. We demonstrated models capable of restoring spectral information, even in areas with highly varying intensities, while restoring the latent sharp image.
\smallskip

\href{https://abarton51.github.io/CS_4641_Project/tabs/final_report.html}{\projectHeadingWithDate{Exploring Music Classification}{CS 4641, Machine Learning, Georgia Tech |}{Aug 2023 - Dec 2023}{}{}}
Led a group project on music classification on two distinct datasets for two different classes - composers and genres. Created a framework in Python for audio data processing, dimensionality reduction, and supervised learning methods such as convolutional neural networks and gradient-boosted trees. Created a Jekyll-powered website for the project.
\smallskip

\href{https://github.com/abarton51/cnn-bird-clf/tree/main}{\projectHeadingWithDate{Bird classification with CNNs}{MATH 4210, Math of Data Science, Georgia Tech |}{Jan. 2023 - May 2023}}
%Completed a project on a colored image multi-class classification task using convolutional neural networks (CNNs). Implemented 3 distinct CNNs in Python using TensorFlow on a dataset of over 15,000 RGB images of birds belonging to 100 separate classes. Concluded with an analysis of performance across various machine learning methods and optimization techniques. 
Explored 3 distinct Convolutional Neural Network (CNN) models on a multi-class image classification task in Python using Keras. The dataset consisted of approximately 88,000 bird images belonging to 515 different classes/species. % The full dataset consisted of approximately 88,000 bird images belonging to 515 distinct species. Exploration of techniques were conducted on a subset of 100 species and 16,000 images.
%Concluded with an analysis of results and performance across various hyperparameters, optimization techniques, and other machine learning methods.
\smallskip

\href{https://github.com/abarton51/simplerw_and_pathenum}{\projectHeadingWithDate{Simple Random Walks and Enumeration}{MATH 3235, Probability Theory, Georgia Tech |}{Oct. 2022 - Dec. 2022}}
Wrote a paper surveying simple random walks under probabilistic and combinatorial perspectives. Showcased the Gambler's Ruin problem and multivariate generating functions as a path enumeration technique.
\smallskip

%\projectHeadingWithDate{Dyck path enumeration}{Georgia Tech School of Mathematics |}{Sep. 2022 - Dec. 2022}
%Participated in the Directed Reading Program hosted by the School of Math at Georgia Institute of Technology. I studied enumerative combinatorics related to Dyck paths under the guidance of a postdoctoral researcher.
%\smallskip


%\projectHeadingWithDate{Warrior Scholar Project}{Warrior Scholar Project at Yale University |}{August 2020}
%College education preparation program for enlisted veterans transitioning to student life hosted by Yale University.
\sectionsep
\vspace{-12pt}

%\section{Conferences and Presentations}
%\vspace{-4pt}
%\begin{skillList}
 %   \singleItem{BEER 2023}{Attended the 2023 International Symposium on Biomathematics and Ecology Education and Research as the lead} \\
  %  \singleItem{}{speaker for an oral research presentation.} \\
   % \singleItem{2023 NCSU Undergraduate Research Symposium}{Led a poster presentation session for research done with an REU program.}
%\end{skillList}

\section{Activities}
\vspace{-6pt}
\begin{skillList}
    %\singleItem{Data Science @ GT}{Member of KaggleCLEF team. We compete in Kaggle competitions and practice and learn about data science.} \\
    \singleItem{\href{https://www.vip.gatech.edu/teams/vxv}{Vertically Integrated Program (VIP)}}{Member of GA Tech FinTech Lab's VIP for "Machine Learning for Financial Markets".} \\
    \singleItem{Directed Reading Program}{Studied enumerative combinatorics under the guidance of a postdoctoral researcher over Fall 2022.} \\
    \singleItem{Georgia Tech Cycling Club}{Recreational mountain biking and road cycling.}
    %\singleItem{Georgia Tech Trading Club:}{General member; Weekly hour long meetings to discuss and explore recent news and trading strategies.} \\
   % \singleItem{Georgia Tech Trading Club:}{Weekly meetings to discuss mathematics and solve problems similar to Putnam style} \\
    %\singleItem{}{ math competitions.} \\
    %\singleItem{Georgia Tech Blockchain Club:}{General Member}%; We meet weekly, attend seminars, and discuss blockchain technology from} \\
    %\singleItem{}{perspectives of development, theory, investing, decentralized finance, and Layer-1 technologies.} 
\end{skillList}

\iffalse
%--------------------------------------------------------------
%     Skills
%--------------------------------------------------------------
\section{Skills}
\begin{skillList}
    %\singleItem{Mathematics:}{Probability and Statistics, LaTeX}\\
    \singleItem{Programming Languages:}{Python, Java, C}\\
    \singleItem{Programming Packages, Libraries, etc.}{Pandas, TensorFlow, Keras, Numpy, Matplotlib, Seaborn} \\
    \singleItem{Programming Concepts}{Artificial Intelligence, Data Science, Data structures, Algorithms, Object-oriented programming}\\
    \singleItem{Other:}{Enthusiasm; Collaboration; Initiative; Communication; Analytical}
\end{skillList}
\fi
\vspace{-2pt}
%this needs some editing but overall not bad
\iffalse
\section{Security Clearance}
\singleItem{Secret-level clearance (2016 - 2020),}{currently inactive}
\fi
\iffalse

\vspace{1pt}

\section{Awards}
\awards{Academic Awards}{Notable Military Awards}{Dean's List \hfill John H. and Susan S. Traendly Scholar \\ Zell Miller Scholar}{Certificate of Commendation \hfill Meritorious Promotion \\ Good Conduct Medal \hfill Arctic Service Medal}
\fi
\vspace{0pt}
\section{Awards}
\vspace{-4pt}
\awards{Academic Awards}{Notable Military Awards}%
{John and Susan Traendly Scholar}%
{Zell Miller Scholar}%
{Dean's List}%
{Certificate of Commendation}%
{Meritorious Promotion}%
{Good Conduct Medal}%{Arctic Service Medal}

\end{document}